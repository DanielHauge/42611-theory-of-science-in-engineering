\section*{Fase 5}

\subsection*{3 Vigtige positive påvirkninger}
\begin{itemize}
    \item \textcolor{green}{$\uparrow \uparrow$} \textbf{10: Reduced inequalities}\\
    Streaming tilbyder højere tilgængelighed samt stører mængder til en lavere pris, hvilket gør musik mere tilgængeligt for flere mennesker trods indkomst forskelle.

    \item \textcolor{green}{$\uparrow $} \textbf{12: Responsible consumption and production}\\
    Med udfasningen af CD'er og LP plader vil behovet for PVC og Polycarbonat plast formindskes, hvilket resultere i mindrer produktion af miljøskadelige materialer.
    
    \item \textcolor{green}{$\uparrow $} \textbf{14: Life below water}\\
    Som effekt af mindre plast produktion vil der akkumuleres mindre plast i havene, hvilket er positivt for livet i vandet.

\end{itemize}

\subsection*{3 Vigtige negative påvirkninger}
\begin{itemize}
    \item \textcolor{red}{$\downarrow \downarrow $} \textbf{13: Climate action}\\
    Som resultat af højere konsumering af online streaming tjenester, vil forbruget af data centrer og medfølgende klima omkostninger stige.
    Disse klima omkostninger omfatter både brug af nedkøling og potentielt brug af forurenende energi kilder.

    \item \textcolor{red}{$\downarrow $} \textbf{6: Clean water and sanitation}\\
    En forøgelse af behovet for elektronik til servere og nedkøling, stiger behovet for minedrift efter råmaterialer. Denne minedrift kan medfører foruerening af grundvandet, der kompromisere rent vand.

    \item \textcolor{red}{$\downarrow $} \textbf{15: Life on land}\\
    Minedrift har også en negativt indvirkning på biodiversitet som eksempelvis et resultat af skovrydning, jordforsuring og erosion.
\end{itemize}

En trade-off der kan betragtes er at skrue på hvilke energi kilder der bruges i systemet.
Man kunne vælge at sætte strænge krav til at alt energi i systemet er fra vedvarende grønne kilder, det vil forbedre de negative påvirkninger til mål 13.
Dog vil prisen stige og derfor forringe de positive påvirkninger til mål 10. 
En forbedring til den negative påvirkningen til mål 15, kunne være at udelukkende vælge minedrift eller opfør ny minedrift som overholder sikkerheds foranstaltninger og indføre en ny process der restorer det naturlige miljø.
Igen ville disse krav på minedrift også øge prisen og forringe de positive påvirkninger til mål 10.

Musik kan anskues som en luksus vare og distribution udelukkende med streaming er ikke et sandt bæredygtigt system. 
Dog er vil ændringen ikke medføre kolosale påvirkninger set i det store billed. Givet en uændret energi produktion i 2030, ville ændringen ikke være forsvarlig fra et bæredygtigt perspektiv.
Derfor anbefales politikere til at indføre lovgivning der gør bæredygtigte energikilder mere rentable. 
Hvis omkostninger på bæredygtigte energikilder nærmer sig eller falder under andre energi kilder, vil det skabe et højt incitament for at bruge bæredygtigte energikilder.
Virksomheder anbefales til at vælge bæredygtigte energikilder, hvilket kan blive et nemmere valg at tage, hvis prisen nærmer sig eller falder under andre kilder. 
Virksomheder anbefales også til at have strængere regler for leverandør kæden af deres udstyr helt ned til minedriften, så der eventuel negative påvirkninger kan mindskes.
Brugere anbefales til at vælge streaming tjenester der anvender bæredygtigte energikilder. 
Brugere anbefales også til at bortskaffe deres tidligere brugte CD'er og LP plader på en miljøvenlig måde, som eksempelvis at sende plastikken til genbrug.
