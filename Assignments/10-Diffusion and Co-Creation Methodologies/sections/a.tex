\section*{Assignment 10a: Diffusion}
For this assignment, autonomous vehicle technology as described in assignment 9a will be used. 
The primary adapters of the technology for this discussion will be consumers or regular drivers who need to get from point a to b, like to and from work.

It is the individual decision that will be focused on, with only 42\% willingness among customers to use autonomous vehicles \href{https://www.statista.com/statistics/1231184/willingness-among-customers-worldwide-use-autonomous-vehicles-by-means-of-transport/}{[1, statista link]}.

\subsection*{Social System}
The social system is people worldwide 18 years or older.

It is expected that humans are in control of technology, in the most direct meaning possible. 
This social norm is hindering the diffusion of the concerns for this technology. 

The following consequences could occur as a result of diffusion.
\begin{itemize}
    \item An accelerated adaptation of technology. This could be considered a desirable anticipated direct consequence. 
    \item Reduced death and injury from traffic accidents by increased adaptation. This could be considered a desirable anticipated indirect consequence.
    \item Over trusting of autonomous technologies, as to neglect the remaining potential safety issues of this and other autonomous technologies. This could be considered an undesirable unanticipated direct consequence.
\end{itemize}

\subsection*{Diffusion ammunition}
The following characteristics could be leveraged for diffusion.
\begin{itemize}
    \item Relative advantage.\\
    Statistics of death and injury-related accidents of non-autonomous vehicles could be leveraged to showcase how much safer autonomous transportation would be.
    \item Compatibility.\\
    Demonstration of the capabilities of autonomous vehicles to operate in current transit systems.
    \item Complexity.\\
    Operating an autonomous vehicle would not require a traditional driver's license to operate. Education in operating a self-driving car would be so much easier, a formal "educational" licensing procedure would likely not be needed. 
    \item Trialability.\\
    A very likely autonomous taxi service would make trying out an autonomous vehicle very possible.
    \item Observability.\\
    Autonomous vehicles would be driving on public roads and be very visible. If autonomous vehicles were made very distinguishable from self-driving cars, it would leverage the effects of observability.
\end{itemize}

\subsection*{Channels}
The advantage of mass communication is a broader reach, but it comes at the cost of decreased trust.
The advantage of more exclusive personal communication is higher trust but at the cost of very little reach. 

The adaptors could be considered to be almost every person on earth, as almost everyone require some kind of movement over longer distances.
Channels that would be effective at targeting a broad audience, could be mass communication through official news.
Another channel that could also be effective is the emerging social media platforms or search engines online. 
Results from search engines are trusted by 2\% more people than traditional media at 59\% percent of people, according to Statista \href{https://www.statista.com/statistics/381455/most-trusted-sources-of-news-and-info-worldwide/}{[2, statista link]}.  

\subsection*{Adaptors}
The early adaptors would likely be those with money and a great need to travel long distances regularly.
A white-collar worker would very likely find it enticing to be able to work productively during travel to and from work.

The late adaptors would likely be those with little need to travel or lack of money.
A store cashier that lives 5 walking minutes away from work would likely not hurry to purchase or become a heavy "taxi-user" of autonomous vehicles.
