\section*{Face mask studies}
Using scholar.google.com, with searches on face mask studies related to corona, it reveals about 30.000-35.000 results depending on search string permutations.
The two chosen countries to study are Denmark and Italy.

\section*{Science driven policies}
\subsection*{Danish health authority}
When visiting the website of the national health authority of Denmark www.ssi.dk and www.sst.dk \href{https://www.ssi.dk}{[1, Statens serum institut link]}, \href{https://www.sst.dk}{[2, Sundhedsstyrelsen link]} , there are many easy-to-digest articles and recommendations about face masks.
Many of the articles have links to data, surveys, research and reports primarily done internally. The following links have been used in assessments: 
\href{https://covid19.ssi.dk/hygiejne/borgere}{[3, Hygiejne for borgere]},
\href{https://www.ssi.dk/aktuelt/nyheder/2021/brug-af-ffp2-3-masker-i-sundhedssektoren}{[4, Brug af FFP2/3-masker i sundhedssektoren]},
\href{https://www.sst.dk/da/corona/Forebyg-smitte/Mundbind-og-barrierer}{[5, Brug af mundbind]}, 
\href{https://covid19.ssi.dk/analyser-og-prognoser/modelberegninger}{[6, Modelberegninger ifm. covid-19]}, 
\href{https://covid19.ssi.dk/analyser-og-prognoser/risikovurderinger}{[7, Risikovurderinger ifm. covid-19]}.

The authorities are doing data collection, analysis and research themselves. In other words, they are drawing on the science that they do themselves.
They do not only draw on science they do themselves, but also from other broadly accepted research-producing bodies.
For example, for the risk assesment report from \href{https://covid19.ssi.dk/-/media/arkiv/subsites/covid19/risikovurderinger/2022/opdatering-af-risikovurdering--ba4_ba5_ba2121---21062022.pdf?la=da}{[8,Risikovurderinger ifm. covid-19 PDF Link]},
Studies produced by the center for disease control \href{https://covid.cdc.gov/covid-data-tracker/#variant-proportions}{[9, CDC Study]} are drawn upon as well.

Although it can be argued that danish health authorities are drawing upon science to construct their recommendations, there are cases shown where science is drawn upon to a lesser extent.
In an article by DR \href{https://www.dr.dk/nyheder/viden/kroppen/nye-krav-om-mundbind-anbefalinger-hviler-paa-et-spinkelt-grundlag}{[10, DR Article Link]}, it is demonstrated how Danish health authorities would draw upon less recognized scientific evidens, even when more recognized research is available.
With the recommendations being made with the neglect of a rather recognized study, it should be evident that recommendations are being made without more than just a scientific basis.

It could be speculated that Danish authorities prefer to trust research made by themselves or other danish actors instead of studies of foreign origin.


\subsection*{Italian health authority}
The health authority of Italy has like the Danish counterpart various information and recommendations for face masks. 
When visiting www.iss.it \href{https://www.iss.it}{[11, Istituto Superiore di Sanità]}, the primary recommendations for face mask use are available:
\href{https://www.iss.it/web/iss-en/home?p_p_id=com_liferay_portal_search_web_portlet_SearchPortlet&p_p_lifecycle=0&p_p_state=maximized&p_p_mode=view&_com_liferay_portal_search_web_portlet_SearchPortlet_mvcPath=%2Fview_content.jsp&_com_liferay_portal_search_web_portlet_SearchPortlet_assetEntryId=5537109&_com_liferay_portal_search_web_portlet_SearchPortlet_type=content&p_l_back_url=https%3A%2F%2Fwww.iss.it%2Fweb%2Fiss-en%2Fhome%3Fp_p_id%3Dcom_liferay_portal_search_web_portlet_SearchPortlet%26p_p_lifecycle%3D0%26p_p_state%3Dmaximized%26p_p_mode%3Dview%26_com_liferay_portal_search_web_portlet_SearchPortlet_redirect%3Dhttps%253A%252F%252Fwww.iss.it%252Fweb%252Fiss-en%252Fhome%253Fp_p_id%253Dcom_liferay_portal_search_web_portlet_SearchPortlet%2526p_p_lifecycle%253D0%2526p_p_state%253Dnormal%2526p_p_mode%253Dview%26_com_liferay_portal_search_web_portlet_SearchPortlet_mvcPath%3D%252Fsearch.jsp%26_com_liferay_portal_search_web_portlet_SearchPortlet_keywords%3DFace%2Bmasks%26_com_liferay_portal_search_web_portlet_SearchPortlet_formDate%3D1660814304777%26_com_liferay_portal_search_web_portlet_SearchPortlet_scope%3Dthis-site}{[12, Face masks iss recommendations link]}

There is substantially less information, references, data collection, studies etc done by the Institut compared to the Danish counterpart.
In most of their recommendations, it is hard to assess whether the basis is from scientific studies.
Although a scientific basis is not as transparent as shown in danish authority recommendations, there is evidence that scientific studies are drawn upon through other research-producing bodies.
The prevention strategy \href{https://www.iss.it/en/web/guest/home?p_p_id=com_liferay_portal_search_web_portlet_SearchPortlet&p_p_lifecycle=0&p_p_state=maximized&p_p_mode=view&_com_liferay_portal_search_web_portlet_SearchPortlet_mvcPath=%2Fview_content.jsp&_com_liferay_portal_search_web_portlet_SearchPortlet_assetEntryId=5498191&_com_liferay_portal_search_web_portlet_SearchPortlet_type=document&p_l_back_url=https%3A%2F%2Fwww.iss.it%2Fen%2Fweb%2Fguest%2Fhome%3Fp_p_id%3Dcom_liferay_portal_search_web_portlet_SearchPortlet%26p_p_lifecycle%3D0%26p_p_state%3Dmaximized%26p_p_mode%3Dview%26_com_liferay_portal_search_web_portlet_SearchPortlet_redirect%3Dhttps%253A%252F%252Fwww.iss.it%252Fen%252Fweb%252Fguest%252Fhome%253Fp_p_id%253Dcom_liferay_portal_search_web_portlet_SearchPortlet%2526p_p_lifecycle%253D0%2526p_p_state%253Dnormal%2526p_p_mode%253Dview%26_com_liferay_portal_search_web_portlet_SearchPortlet_mvcPath%3D%252Fsearch.jsp%26_com_liferay_portal_search_web_portlet_SearchPortlet_keywords%3Dface%2Bmasks%26_com_liferay_portal_search_web_portlet_SearchPortlet_formDate%3D1660826067982%26_com_liferay_portal_search_web_portlet_SearchPortlet_scope%3Dthis-site}{[13, strategy pdf link]} is using references from recognized international organizations like the world health organization \href{https://www.who.int/}{[14, WHO website]} and the european center for diseases prevention and control \href{https://www.ecdc.europa.eu/en}{[15, ECDC website]}.




\section*{Policy drivers}
As a general policy driver, policies are made to appease the public for powerholders to stay in power.
Regardless of what the public thinks they want, positive results still have to be shown, which is why there are quite a lot of decisions that are science-based to more reliably produce the desired outcome.
Depending on the country, the public might be more or less aligned with science and is therefore also a justification to use studies as a basis for decisions.
So when studies are not available, the policies that are mobilized, are dictated by the culture, ethics, morals and social values that the people hold.
If less popular unscientific policies are put in effect, the chances for another party to be elected increases, which is a pretty big incentive to avoid it. 

\section*{Public trust}
Denmark has shown a long history of transparency. 
So long and transparent in fact, that Denmark shares first place with 4 other countries in being the least corrupt country in the world as of 2021 \href{https://www.transparency.org/en/cpi/2021}{[15, transparency index]}. 
Denmark is very progressive in many areas such as gender equality, social security, green transition and many others. Denmark can even be considered a role model to many other nations \href{https://d.newsweek.com/en/file/462486/denmark-report.pdf}{[16, Denmark as rolemodel article]}.
Whenever corruption does occur, it is punished very harshly and reported very widely and with an abundance of shame. Danish corruption is handled very differently in contrast to what it is handled in many other places on earth.
With a relatively long history of low corruption, high transparency, high scientific policy basis etc, it could be speculated that the Danish authorities have gained a high level of trust. 

Italy is ranked 42 out of the total 180 that is ranked. Rank 42 should not be considered bad, but still worse than the Danish counterpart.
Generally, the trust in the Italian government can be considered on shaky grounds. According to an article by quartz \href{https://www.scu.edu/ethics/focus-areas/journalism-and-media-ethics/resources/the-trust-issue-in-italy/}{[17, QZ article]} and a study from Santa Clara University \href{https://www.scu.edu/ethics/focus-areas/journalism-and-media-ethics/resources/the-trust-issue-in-italy/}{[18, SCU Study]}.
A history of corruption without proper resolution or closure can be observed in Italy, and this has implications for the trust of the public.
It could be speculated that the Italian health authorities have less trust in their recommendations because of a history of corruption in the government and the lack of a transparent scientific basis.

\section*{Mask issues}
Quite a lot of the studies that indicate disease transmission reduction through face masks are done in a clinical setting. 
Furthermore, recommending face masks to all does include those with breathing difficulties like those with astma.
Wearing face masks is very common in clinical operations, and is usually worn by professional doctors or surgeons etc.
Without proper training in face mask-wearing, the positive effects could be negligible and not worth it in total when considering potentially negative effects.
This makes the science on which the recommendations are based a little uncertain. 
With the uncertainty of the basis for the recommendations, citizen science projects could be carried out to either confirm or reject the science and potentially strengthen or weaken the recommendations.
An example of a citizen science project could be to try to map touch with invisible markers that are only visible under UV light.
The project could potentially show normal people without proper training unintentionally touching, rubbing and disturbing the mouth area enough to eliminate the positive effects shown in clinical studies.




