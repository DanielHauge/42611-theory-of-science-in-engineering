\section*{Assignment 3b: Disciplines and models} 
Disclaimer: I'm working as a group that consists only of me, however for this assignment, I've decided to imagine a fictitious scenario of a team with more members.

The team of which the technology will be developed is interdisciplinary.
The primary disciplines of the team are within the fields: Computer science and physics. The disciplines of the team members would not be mutually exclusive.

\begin{itemize}
    \item \textbf{Physics}
    Physicians would contribute with theory, understanding and truth of the physical phenomenon that the technology relies on. 
    In this team, this discipline aligns more with that of a thinker/scientist, in an attempt to comprehend and seek truths.
    In regards to the problems of the experiment, the physician could provide a solution to the problem of influence/interference from the environment.
    Instantaneous communication relies on an understanding of a "yet-to-be-known" physical phenomenon, which is why physics will contribute greatly in regards to accuratly understand the necessary truths to form a theory.

    \item \textbf{Computer science}
    Computer engineers would contribute with a model in which the technology would work. The model is produced based on the theory that members of the physics discipline.
    This includes designing and implementing the physical device, protocols, standards, software libraries and more that the technology relies on.
    In this team, this discipline aligns more with that of a maker/engineer, in an attempt to transform the world by putting the theory into actionable practice.
    In terms of the problems of the experiment, one of the identified problems the computer engineer would help solve could be the imprecise or uncertain measurements.
    Instantaneous communication is solving a problem within the field of computer science, which is why the discipline of computer science will contribute greatly in regards to safe, dependable and practical transformation.
\end{itemize}

Collaborating between different disciplines can add substantial value to the solutions that are provided. 
In this case, it could be argued that collaboration is not only value-adding but necessary.
The technology manifests as a device that is heavily rooted in computer science but relies heavily on a physical phenomenon, which is why both disciplines are deemed necessary.

One of the strengths of working with a small team is that the technology has a consolidated design and output. The solutions can be optimized in a specialized manner, as to accommodate the perspective of the computer scientist completely. 
One of the weaknesses is the lack of differing perspectives. 
The technology could result in unwanted sustainability issues, for example, environmental hazards that might be caused by the use of the technology. 
The solution could be more relevant with the addition of more perspectives, like from the perspective of an ecologist.
