\section*{Assignment 4a: Ethical dilemmas in research}
Ethical dilemmas are hard, as ethical commitments to all stakeholders should be fulfilled.
If for some reason full ethical commitment to all is not possible, for example with a dilemma, stakeholders should be prioritized based on the situation.

Although it should depend on the situation, my opinion sways very often towards the greatest commitment to the public. 
The reason is that the public often involves a larger amount of people. If integrity breaches are inevitable, I think they should affect the least amount of people with the least ulterior motives.

Despite the funder being the reason for enabling a research project, I often feel the least ethical commitment to the funder.
I find the funder to have the most amount of ulterior motives, such that research projects are only funded with a wish for specific results to be revealed.

These opinions are a product of an uncertain trend that I've observed the most, therefor the situation should be emphasized a lot when deciding upon prioritizing ethical commitment.
Different scenarios would produce different prioritizations based on the situation and severity of the integrity breach. 
But bottom line is, that ethical commitment neglect should be avoided if possible.

The following fictitious scenario capture how I feel a greater commitment to the public and the least to the funder.  

\subsection*{Scenario}
An online social media platform company has funded a research project, to quantify the effects of its advertisement services. 
The company expects substantially noticeable sales increases from its advertisement. however, the results of the research reveal negligible positive effects.
Publishing the results will very likely have a damaging effect on the social media platform, hence a wish to withhold or diffuse the results is made.

Without the funding, the results would never have been produced, should the results be withheld or diffused?

In this case, the public should be prioritized. To uphold research integrity, the researcher has the right and the duty to publish their results to the scientific community, relevant professionals and the public.
Withholding or diffusing the results breaks the integrity by not having transparency or honesty with results and conclusions.

\newpage