\section*{Assignment 4b: Ethics framework} 

\subsection*{Emerging technology}
Text-To-Speech machine learning (ML) technology is steadily becoming more advanced, so advanced in fact that an open source project is now working to imitate an actual person's voice.
The project works of Dan Ruta: \href{https://github.com/DanRuta/xva-trainer}{xVA-Trainer} and \href{https://github.com/DanRuta/xVA-Synth}{xVA-Synth} are software projects that can train machine learning models which can then replicate the voice reminiscent of the voice which the model was trained with.

\subsection*{Benefits}
The main benefit of the technology is to enable quickly produced and affordable voice lines for modded games like Elder Scrolls: Skyrim and Fallout. 
Skyrim and Fallout are story-driven games with an abundance of dialog in the form of text. The technology enables mods that replace the textual dialog with voice-lined dialog.
Besides the production of voice line audio in modded games, it can potentially be used in regular small to large-scale game development. 
A voice model can be trained once, and then used to generate voice lines.
This could potentially cut down voice line development costs, especially for the types of games with an abundance of dialog.
In the same manner, audiobooks, presentation videos and other types of content like youtube videos could also potentially benefit from this technology.
These benefits are deemed pretty substantial and very likely to happen. 

\subsection*{Extrinsic concerns}
With the ability to imitate a voice, a potentially concerning dual-use arises. 
It would become easier to record and train a voice model from a person unwillingly, which could then be used to generate speech with that person's voice.
This dual-use could lead to identity theft issues. 

With the advancement of naturally sounding text-to-speech ML technology, laws related to verbal speech should be revisited.
Verbal agreements are binding and could become easy to fabricate with the technology.

There is also an indirect environmental concern related to advancement in text-to-speech ML. There is a pretty hefty computing cost involved when training a voice model. 
A rough estimate for a model training is as follows: dataset with 800 voice lines averaging 5 seconds, it takes approximately 3 days of full throttle for an RTX 2080TI NVidia graphics card. 
An RTX 2080TI NVidia graphics card consume about 277W \href{https://www.tomshardware.com/reviews/nvidia-geforce-rtx-2080-ti-founders-edition,5805-10.html}{[1]}.
$ 277W*72h = 19944 Wh = 71.79 MJ $

If the energy used to power model training is not exclusively from sustainable green sources, it is a reason for concern. 
Although 71.79 MJ can be considered insignificant, in case the technology becomes widely used then the effect is compounded.

\subsection*{Power analysis}
The technology empowers criminals that intend to use the technology with malicious intent.
As a result of a new method for identity theft, it disempowers the victims of said malicious intent.
The older generation is also likely to be disempowered, as they might not be accustomed to or familiar with technological advancement, and might not be able to identify a familiar fake voice from the real one. 
Game developers will be empowered, as the development of voice dialog will be cheaper and faster.
Content creators like YouTubers and/or film creators will be empowered as they too will be able to fabricate voice lines for their content.

