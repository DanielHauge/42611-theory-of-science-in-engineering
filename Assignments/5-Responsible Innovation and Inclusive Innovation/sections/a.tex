\section*{Assignment 5a: Responsible Innovation}

\subsection*{Contested innovation: Human Genetic Engineering}
With the advancement of CRISPR/Cas9 technology, human genome engineering becomes more feasible. With the control of the human genome, disadvantageous traits can be replaced with advantageous ones.
Human genome engineering has applications in eliminating many genetic-related diseases, but may also be used to select desirable traits in the non-health category, like hair color et cetera. 
The advancement and implementation of human genome engineering could improve the overall sustainable health goals, and also solve potential future genetic degradation problems.
It is however not possible to select desirable traits in newborns, this is because the technology is not developed enough, but also partly due to being contested. 
The application of the technology can be considered playing god, which is an intrinsic ethical concern and the main reason for being contested. 
Although not the main reason for contention, there are also health and well-being related to human gene technology. 
As with any intervention of human biology, there are risks involved, as something could go wrong. 
There is also equality concerns related to human genetic engineering.
In case not all people would have access to the technology, or access to some desirable traits et cetera.
As the contention stems from religious reasons, increased responsibility in human genome engineering innovation is unlikely to improve the situation. 
But if we disregard the religious reasons for contention, we could alleviate concerns by escalating to more strict standards with the \href{https://www.wma.net/what-we-do/medical-ethics/declaration-of-helsinki/}{[1,"Declaration of Helsinki"]} instead of for example the medical version of GxP: \href{https://www.fda.gov/about-fda/center-drug-evaluation-and-research-cder/good-clinical-practice}{[2,GCP]}. 
Establishing regulations for fair use and access to all could help eliminate the concern of inequality. 
With these examples, human genetic engineering could be made a more responsible innovation.

\subsection*{Neglected areas: Global warming}
Regardless of being aware of the negative consequences and impacts of our non-sustainable energy consumption and other climate deteriorating actions, we have yet to solve or even get close to approaching a projectory to meet our goals.
Effects of our lacking effort in climate action goals are resulting in global warming, which potentially risks all life on the planet in the future.
Although it can be argued that initiatives have been taken to improve the situation, it can also be argued that it is still nowhere sufficient enough.
Responsible innovation can improve the climate action situation. Impact assessments like emission assessments and standardizing sustainable energy sources and transport methods.
The situation can also be improved by ilegalizing the use of substances that are polluting like hydrofluorocarbons used in refrigerators and air conditioners.
Hydrofluorocarbons are 3790 times more damaging to the climate than carbon dioxide over a 20-year period \href{https://www.ccacoalition.org/en/slcps/hydrofluorocarbons-hfcs}{[3, ccacoalition hfcs]}.
Another example could be to increase government research funding on for example Power-To-X, Renewable Green energy or other projects that seek to improve the climate.

\newpage