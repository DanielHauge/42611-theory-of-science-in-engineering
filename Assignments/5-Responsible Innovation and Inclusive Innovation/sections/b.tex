\section*{Assignment 5b: Inclusive Innovation} 

Innovation should be made inclusive, such that the outcome of the innovation will provide a positive impact without neglecting to consider the possible negative impacts on others.
The problem that the innovation seeks to solve, should solve the problem for all those who experience struggle with it, while also avoiding causing problems elsewhere. 

Places with medium to high income often have a high living standard compared to low-income places. 
Although low-income places are capable and have plenty of innovation, it is not uncommon to spend an overwhelming amount of time working just satisfying basic needs like food, shelter, clean water etc. 
Higher-income places have more time to innovate, and with neglect towards lower-income places, the innovation usually targets the problems that are observed by oneself rather than everyone else.
Because of the skewed innovation target and available time allocated toward innovation, a large amount of innovation is therefore targeting these higher-income places.

One of the challenges is that problems differ vastly from where in the world that is considered. 
Innovations seek to solve problems, but with the amount of complexity, development and equality disparity that exists in the world, it can be hard to solve a problem inclusively.

Another challenge of achieving inclusive innovation is to overcome an instinctive human behavior, that humans often limit altruism to the "tribe"/ closer circle.
Most cannot or will not afford to think of anyone else than themselves and their closest, and in conjunction with capitalistic thinking, this leads to innovation that in simple terms have to "sell".
With a goal of economic growth and self-preservation, who better to target than those with higher resources/income?

My study line is software technology, and with the fact that an overwhelmingly large part of the world does not have access to computers or the internet, it is very seldom that I observe or think of inclusive innovations. 
It is seldom to see or think of innovations in computer science as pro-poor, especially if we consider the very poor in third-world countries.
The lives of people from low-income third-world countries are vastly different with vastly different problems than that of the more developed parts.
Although very different, examples of inclusive computer science innovations that also include third-world countries do exist.
An example of computer science innovation that can be considered inclusive, would be the SpaceX Star-link project that aims to provide high-speed, low-latency internet to rural locations across the globe \href{https://www.starlink.com/}{[4, starlink]}. 

Another example of inclusive innovations in computer science could be the many accessibility features like colorblind mode, screen magnifier, etc. 
Although accessibility features like those mentioned could be considered an innovation by themselves with the sole purpose to include an excluded group.
People without a colorblind diagnosis would usually not use a colorblind feature but will solve the color related computer problem for all regardless of colorblindness or not.