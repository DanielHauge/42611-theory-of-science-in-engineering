\section*{Assignment 6a: water closets}

\subsection*{Debate}
The theory of miasma was the main theory of medical science in the 19th century. In the theory of miasma, it is believed that odors, smells and noxious air was the primary driver of diseases.
As a result of increasingly crowded urban development in conjunction with an excrement disposal method that increases miasma, sanitary and health conditions decrease.
Not only are the consequences of a less than ideal excrement disposal method visible by the unsightly dungheaps and pigsties along the streets, but especially also because of health deterioration of the population.

The best argument for water closets is sanitary reasons. The sanitary conditions are way too bad and need conditions to be improved.

Development to provide everyone with sanitary solutions is costly, so costly that it is not affordable for everyone.
Poor people have very little money, so little most people were generally in debt at end of the week from living expenses.
The people with the resources to provide are not altruistic enough to provide for the poor.

The best argument against water closets is monetary reasons. It is too expensive for anyone but the rich to afford.

\subsection*{The 'best' technology}
Technologies serve to provide the means to satisfy the needs of people. The needs can originate from problems in all areas of life, related to things such as health, social, survival et cetera.
As life is very complicated, the solutions to the problems that arise are also very complicated. The complications trickle up the chain to the technologies that are developed.
From an objective/technical perspective, it is very difficult if not impossible to determine which technology is the best one. Technology affects so many things both directly and indirectly.
If a hypothetical extremely simple technology is replaced by another slightly better extremely simple technology, then it could be argued that out of the two, one is the best.
Measuring the full effects of a specific technology is very difficult if not impossible in a real scenario, as the world and the technologies within are so complicated.

From a technical perspective, it is also hard to evaluate and compare the effects of a technology that differs in nature. As an example, let solar panel technology be considered.
Solar panels provide a green source of energy that can produce energy as long as the sun shines, and this improves negative impacts on the climate.
Solar panels do however also require cobalt which is usually mined in third-world countries, and as a result, has shown to increase child labor.
Can we compare an improvement in climate action to the deterioration of social standards, and say which is better objectively?
The 'best' or better technology is a subjective matter. We as humans, put value and severity on solutions and problems, and from that decide what is more important in the given context.
Values and severity are decided by culture, customs, morals, ethics and the agreed-upon facts and knowledge about the world, which all change over time. 
The 'best' technology might not be the 'best' technology tomorrow, perhaps because of new knowledge from research, a social paradigm shift or perhaps a war breaks out.


\newpage