\section*{Assignment 6b: Revolution} 
Disclaimer: As I work alone, I will only account for my own experience, but attempt to reflect on more general notions of what I believe is true for others. 

In my lifetime, the most impactful domestic technology changes I have observed, have been in computer systems.
Although throughout the years, the access to these technologies has only decreased, with parents divorcing and moving out.
I have not observed any change in relations as an effect of domestic technology changes, at least not ones that are likely caused by changes to domestic technologies.
I believe this to be true for most people in their mid-twenties or younger.

In regards to resonating with the division of labor and social relation structures of Singapore and America, the experiences I've accumulated can barely resonate.
I cannot include the relations of my grandparents, as they were unfortunately not alive from a very young age.

In regards to division of labor, my parents both had well-paying and average-hour jobs and did most domestic tasks either together or in turns.
As an example, one of my parents would bring and the other would fetch from school.
One of my parents would dust off while the other would vacuum et cetera.
Although some domestic tasks were primarily carried out by one or the other. 
More dangerous, or strength-demanding tasks were usually done by my father, whereas 'children enrichment' like playing or story reading was done considerably more by my mother.
I believe this to be true for marginally more families in Denmark than elsewhere.
I do however still believe that many families in Denmark have more distinct roles and task division in the family construct.
This slight division of labor is considered a pretty normal practice considering the history and culture of most places including Denmark.
I believe the division is archetypically more divided than what I've observed, as I believe great efforts have been put into the consolidation of the gender roles in Denmark. 

In regards to the relational structures, families of Singapore and America but also Science for that matter, seem to be heavily male dominating.
Male dominance is not something I've personally experienced, almost all relationships I observe are considerably female-dominated, almost to the point of complete submission of the male.
The relationships I observe are those in my family but also relationships of friends.
I've observed the male being ordered around to fulfill the needs of the female, and more often than not surrender to the will of the female.
I do not believe this is typical or representative of the normal relationship structure in Denmark, but not outlandish either.
Denmark is well developed in regards to gender equality, and thus when exceptions to power/gender imbalances are observed it is equally likely to be observed in both directions.
I believe my experiences are an anomaly compared to most people, but as a general opinion, I think Denmark is very progressive when it comes to consolidating the roles, norms etc. of genders.