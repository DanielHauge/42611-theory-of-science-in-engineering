\section*{5G \& IoT}
\subsection*{Potential}
Wireless connection to the internet is a widely used technology, especially for smaller, mobile, handheld or embedded devices. 
A high-speed connection to the internet provides a convenient way to achieve distributed computing, data collection, remote control, monitoring, maintenance prediction and much more.
5G as technology is expected to use 10 times less energy for the same amount of data transfer compared to 4G. 
Data collection is an important tool when it comes to producing assessments, this can be to asses whether an innovation produces the intended effects. 
With sufficient usage data of electronic devices, potential data-driven planning of power production can be realized, which can help promote flexible power usage.
With the advancements in machine learning and AI, systems can be digitalized, controlled remotely and made intelligent and autonomous, which potentially saves resources and energy.
The technology contributes to climate actions primarily in regards to reducing energy expenditures. 
The technology also enables potential flexibilities and optimizations to combat the unstable nature of green energy sources. \href{https://learn.inside.dtu.dk/d2l/le/content/121836/viewContent/455582/View}{[1, https://learn.inside.dtu.dk/d2l/le/content/121836/viewContent/455582/View]}
Although the technology can potentially contribute to climate action, assessments of emissions produced by the technology should not be neglected.
For example, emissions from the result of expanding infrastructure to accommodate 5G should be considered in the total assessment in the attempt to reduce climate gasses. 

\subsection*{Development stage}

Wireless internet connection technology is still widely in use with older generations of the technology, namely 2G, 3G and 4G.
According to Statista, 58\% of the mobile connections worldwide in 2021 were with 4G, 8\% on 5G with the remaining connections being on older generation connections.
5G is forecasted to take up to 25\% of mobile connections in 2025 \href{https://www.statista.com/statistics/1100701/world-mobile-technology-mix/}{[2, https://www.statista.com/statistics/1100701/world-mobile-technology-mix/]}.
The technology readiness level (TRL) as defined by Nasa \& ESA of 5G \& IoT would be TRL 9 considering 5G, as the technology is proven and are currently operational with an 8\% usage worldwide.
In regards to IoT and usage of 5G, many of the applications of IoT and 5G usage are at TRL 8 when 105 companies in Denmark are asked \href{https://learn.inside.dtu.dk/d2l/le/content/121836/viewContent/455582/View}{[1, https://learn.inside.dtu.dk/d2l/le/content/121836/viewContent/455582/View]}. The next step of IoT and usage of 5G is naturally TRL 9, as to realize the untapped potential the technology can bring.

Considering the future of an inevitable new generation of wireless internet connection 6G that might contribute even better to climate action, the TRL can be considered to be at TRL 2. 
Recent academic articles have conceptualized a 6G internet connection with additional features involving the use of artificial intelligence \href{https://arxiv.org/abs/1904.11686}{[3,https://arxiv.org/abs/1904.11686]}.
The next step of 6G is naturally TRL 3, which focuses on analysis and experimentation of a proof-of-concept.

\subsection*{Invest}

In regards to reaching full operationality of IoT and 5G usage, there is a large amount of infrastructure work that needs to be done.
5G operates with smaller "cells", which therefore require more towers, 5G coverage is still not developed enough and would be a good place for the government to invest.
In regards to furthering the development of the next generation of wireless connection, tech companies should invest in education and research, as it would help fund research projects to produce proof of concepts, their analysis and experiments thereof.


