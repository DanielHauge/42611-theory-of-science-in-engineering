\section*{Assignment 9a: Controversial technology}

The technology that controversies will be explored is that of self-driving vehicles. 
Self-driving or sometimes also called autonomous vehicles are vehicles that can operate without human intervention. 

With most autonomous technologies, the concern of safety arises. 
Is potential malfunction considered during operation, and if so, will a mitigation procedure be possible and successful in avoiding unintended effects?
Another controversial concern is that the technology could eliminate jobs without producing an equal amount of different jobs, hence increasing unemployment.
Controversial ethical concerns are also at play, as to define the correct actions in dangerous situations.
With a change to digital control, malicious intent like hacking threats also becomes a likely concern.

Information on the subject is very easy to find and gauge, as it is widely discussed, likely due to it being such a potentially impactful technology.
The following search strings have been used on DuckDuckGo and YouTube. (N.B. + indicates conjunctive search. ex. "Autonomous cars Ethics")
\begin{itemize}
    \item Self-driving cars
    \item Autonomous cars
    \item Autonomous trucks
    \item + Controversy
    \item + Debate
    \item + Ethics
    \item + Accidents
    \item + Pros and Cons
    \item + Concerns
    \item + Dilemma
\end{itemize}

A public debate discussing the controversy on the ethical judgment concerns is ongoing, and as the issue is quite complex it has widely different "correct" answers depending on who is asked.
The controversy is best described by the following scenario:

An autonomous car is driving along a 1 lane narrow highspeed cliffside road, Another car is driving in the opposite lane and suddenly a middle-aged adult pedestrian is jumping out onto the road such that a collision cannot be avoided by braking.
There are 3 realistic options available for the autonomous car:
\begin{itemize}
    \item Sway into the oncoming car. This would likely kill or at the very least severely harm occupants of both vehicles.
    \item Collide with the pedestrian. This would likely kill the pedestrian.
    \item Sway away from the road, resulting in falling down the cliff. This would likely kill the occupants of the autonomous car.
\end{itemize}
One could argue that it is the fault of the pedestrian, for putting themselves in a commonly known dangerous situation. 
But what if we replace the middle-aged adult with a child who might not know better? 
Then it could be argued that swaying over the cliff would be preferred, as a means of self-sacrifice for the younger generation.
If the autonomous car occupants are full of children, then it could be argued that one option saves more lives and is therefore preferred et cetera.
This game of adding complexities can be continued for a very long time, which demonstrates how these judgments are so complicated to calculate and define, as there are so many edge cases to consider.
Even if we could manage to consider all possible situations and edge cases, the "correct" answers we find are determined by things like culture, and societal and personal values.
Ethics are not universal and vary with different cultures and societies in the world, there are so many "correct" judgments that have to be accommodated.
As an example of differing judgments, a scenario survey using the \href{https://www.moralmachine.net/}{[Moral Machine, https://www.moralmachine.net/]} reveals that more prosperous countries are less likely to spare jaywalkers than those in poorer regions \href{https://www.forbes.com/sites/jamesmorris/2021/02/13/self-driving-cars-wont-go-mainstream-until-we-solve-this-problem/?sh=2fba5092f3b3}{[Forbes article link]}.
