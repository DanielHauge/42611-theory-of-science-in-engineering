\section*{Assignment 9b: Social Construction of Technology} 

\subsection*{Social groups}
\begin{itemize}
    \item \textbf{Pedestrians}\\
    People who participate in transit in the public areas by walking on sidewalks, crosses, etc.

    \item \textbf{Professional drivers}\\
    People who make a living by driving vehicles, like taxis or trucks.

    \item \textbf{Regular car drivers (against)}\\
    People who participate in transit in the public, intending to transport themselves by car, for example from and to work.

    \item \textbf{Autonomous car occupants (for)}\\
    People who participate in transit in the public, intend to transport themselves by an autonomous car, for example from and to work.
    
\end{itemize}

\subsection*{Meanings and Problems}
\begin{itemize}
    \item \textbf{Pedestrians}\\
    A pedestrian could be insecure about less opportunity for negotiation with an "operator" of an autonomous vehicle.
    A pedestrian might also be afraid to travel due to maybe unfavorable ethical rules programmed into the AI controlling the vehicle.
    A pedestrian could think calmly of autonomous vehicles, as they are very predictable.

    \item \textbf{Professional drivers}\\
    A professional driver would very likely see the technology as a huge threat to their job security.

    \item \textbf{Regular drivers (against)}\\
    A regular driver could be thinking that being driven is boring, and wants to drive themselves.
    A regular driver might also be afraid to drive/be driven due to maybe unfavorable ethical rules programmed into the AI controlling the vehicle.
    A regular driver could be insecure about less opportunity for negotiation with an "operator" of an autonomous vehicle.
    A regular driver could be nervous about not being in control.
    A regular driver might be insecure about the capabilities of the autonomous system.

    \item \textbf{Autonomous car occupants (for)}\\
    An autonomous car occupant might think of autonomous cars as relaxing and convenient, as constant mental attention is not required.
    An autonomous car occupant might think of autonomous cars as efficient and productive, as other things can be done during transit.
    An autonomous car occupant could think calmly of autonomous vehicles, as they are very predictable.

\end{itemize}

\subsection*{Controversial}
The controversies that the technology spark is concerning: Safety issues about malfunctioning systems, like for example interference with sensors or a software bug.
Ethical concerns about what actions the autonomous vehicle should take in less than ideal situations, like when confronted with a situation that requires a choice to be made involving injury and death. 
Automation of many kinds incites job threats to those people that do the tasks manually, and without clear indications that an equal amount of other new jobs become available, it should be considered whether increased unemployment is a good idea.
With the replacement of analog control with digital control, hacking could become a threat and therefore it should be considered whether it is a good idea to digitalize control of vehicles.

The focus will be on the ethical concerns and it is primarily pedestrians and regular drivers that are involved with the controversy.
The pedestrians are involved, as they could be afraid to travel because of fearing being chosen to die instead of occupants of an autonomous vehicle, maybe based on age or some other attribute. 
In the same manner, a regular driver could also inhibit this fear.

\subsection*{Closure suggestion}
If dedicated transit systems are built to be used only by autonomous vehicles, for example, special highways underground.
In addition, limiting all autonomous driving to these systems by law, then it would close or stabilize the controversy.
Without the numerous complicated ethical dilemmas that arise from merging autonomous vehicles, pedestrians and non-autonomous vehicles in a shared transit system, the fears would be lessened.
